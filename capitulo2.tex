%\usepackage[spanish]{babel}
\chapter{Marco conceptual}

\section{Lenguaje no verbal}

Se puede definir el lenguaje no verbal como la herramienta para enviar y recibir mensajes con c\'odigos extralinguisticos como lo son los gestos, posturas corporales, expresiones faciales, tonos de voz, etc.

La comunicaci\'on verbal siempre est\'a influenciada por el lenguaje no verbal \cite{bussinessCommunication}. Por tel\'efono el significado del mensaje es afectado por el tono de voz \cite{bussinessCommunication}. Presencialmente, es afectado por gestos, expresiones faciales, posturas corporales. Incluso al momento de presentar en alguna charla o conferencia, se utilizan elementos audiovisuales, mapas, diagramas, im\'aenes, etc\'etra \cite{bussinessCommunication}.

La comunicaci\'on no verbal se viene desarrollando desde hace muchos a\~nos, incluso antes que el lenguaje verbal naciera como tal. De hecho ciertos animales tambi\'en presentan este tipo de comunicaci\'on [paul ekman, charles darwin].

Los elementos m\'as destacados del lenguaje no verbal son: tonos de voz, espacio, sentido, tiempo y el lenguaje corporal \cite{bussinessCommunication}.



\subsection{Expresiones faciales}
Los estudios de Ekman \cite{EkmanEstudioExpresionesFaciales} revelaron que las expresiones faciales juegan un rol importante al momento de comunicar o sentir una emoci\'on. Cuando se analizaron expresiones del rostro entre distintas culturas, estas mostraron una especie de sincronizaci\'on en sus respectivas emociones representadas \cite{EkmanEstudioExpresionesFaciales}. Por ejemplo, se concluy\'o que japoneses y americanos presentaban acciones faciales similares cuando ve\'ia solos una pel\'icula neutra o una que les provocara estr\'es.

En las investigaciones de Ekman se puede observar que \'este tipo de comunicaci\'on afecta de manera indirecta, pero con gran potencia \cite{EkmanEstudioExpresionesFaciales}. En un caso part\'icular, los resultados obtenidos demostraron que en una tarea de aprendizaje, el profesor castigaba m\'as a los alumnos que comunicaban un rostro m\'as alegre, que a los irritados \cite{EkmanEstudioExpresionesFaciales}. Por otro lado, los estudiantes desarrollan un aprendizaje m\'as efectivo con un profesor que presente m\'as exresiones positivas que negativas durante su ense\~nanza cite{EkmanEstudioExpresionesFaciales}. Los ni\~nos que observaban escenas violentas en la televisi\'on, y exresaban rostros m\'as alegre que tristes, ten\'ian una coducta m\'as agresiva que altruista \cite{EkmanEstudioExpresionesFaciales}.

Todos estos ejemplos demuestran emp\'iricamente que las expresiones faciales y las emociones est\'an relacionadas de cierta manera.


\subsection{Lenguaje corporal}
Se puede enteder por lenguaje corporal aquellos actos o posiciones corp\'oreas que permiten la comunicaci\'on entre dos o m\'as sujetos, mediante un c\'odigo espec\'ifico. Algunos de estos c\'odigos son los g\'estos, y las posturas corporales.
\'este tipo de comunicaci\'on est\'a presente en muchos primates, y en humanos tambi\'en. Sin embargo no se ten\'ia certeza si eran caracter\'isticas innatas o aprendidas. El experimento de japoneses y americanos no convenc\'ia a Ekman, dado que los primeros reaccionaban de manera distinta frente a sus jefes, superiores, u otras personas que a cuando estaban solos. Fue as\'i como Ekman decidi\'o ir a estudiar el lenguaje corporal de una cultura aislada de la civilizaci\'on, en Pap\'ua, Nueva Guinea. Los resultados apoyaron la teor\'ia de Darwin, qui\'en propon\'ia que estos comportamientos eran innatos \cite{EkmanArticulo}.

\subsubsection{Gestos}
Los gestos son movimientos hechos por partes del cuerpo como por ejemplo las manos, dedos, brazos, piernas, cabeza, los cuales pueden ser voluntarios o involuntarios \cite{KurienBodyLanguage}. Se pueden observar en la vida cotidiana, cuando la gente se saluda, se despide, cuando se realizan exposiciones, y as\'i mismo en diversas actividades.
Los brazos tienen multiples interpretaciones sobre lo que se pueda estar comunicando. Por ejemplo al discutir, si el sujeto cruza sus brazos, generalmente representa una especie de rechazo o negaci\'on ante su interlocutor \cite{KurienBodyLanguage}.
El an\'alisis de los gestos puede ser utilizado incluso para  mentiras. Ekman se\~nala que es tremendamente dificil mantener las manos o el cuerpo quieto cuando se siente una emoci\'on intensa "no hay ninguna apariencia m\'as dif\'icil de lograr que la frialdad, neutralidad o falta de emotividad cuando por dentro ocurre lo contrario" explica en su libro "Como detectar mentiras" \cite{EkmanComoDetectarMentiras}.

\subsubsection{Posturas corporales}
Las posturas corporales no se quedan atr\'as, y es que a trav\'es de ellas, tambi\'en se pueden comunicar mensajes. En las salas de clases, las posturas corporales pueden dar a luz qu\'e tan interesados est\'an los alumnos en la ense\~nanza que se les est\'a entregando. 
Los estudios han demostrado que generalmente una persona sentada con su cabeza asintiendo hacia adelante, representa un estado emocional de relajo, por lo cual est\'a m\'as dispuesta a escuchar \cite{KurienBodyLanguage}. En cambio un sujeto con las piernas y brazos cruzados, moviendo su pi\'e, en el mayor de los casos, puede que se deba a que se sienta impaciente, o que est\'a distanciado o en desacuerdo de la discuci\'on \cite{KurienBodyLanguage}.
En el an\'alisis de las diferentes posiciones, Mondloch y su equipo explican que la inferencia de las emociones en base a posturas del cuerpo tiene mayor efectividad cuando se le compara con una emoci\'on diferente o neutral \cite{MondlochRelacionPosturasYGestos}.





Albert Mehrabian estim\'o que al momento de expresar emociones solo un 7\% del lenguaje es verbal, 38\% es sonoro (tonos, matices, volumen) y un 55\% se basa en lenguaje corporal. Por otro lado, afirmaba que en una conversaci\'on cara a cara, un 35\% era comunicaci\'on verbal y un 65\% no verbal \cite{MehrabianNonVerbalCommunication}.

\section{Anal\'itica multimodal del aprendizaje}
\section{Computaci\'on afectiva}
\section{Visi\'on computacional}



%\subsection{Tono de voz}
%\subsection{Espacio}
%\subsection{Sentido}
%\subsection{Tiempo}



