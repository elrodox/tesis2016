\documentclass[12pt,letterpaper]{report}

%Dimensiones de la p�gina
\usepackage[left=3cm,top=3cm,right=2.5cm,bottom=2.5cm]{geometry}
%Sangr�a
\setlength{\parindent}{1cm}

%Numeracion
\pagenumbering{arabic}

%\marginparsep 0pt \textwidth 6in \topmargin 3cm \headsep .5in
\textheight 9.2in \voffset = 0pt \hoffset = 0pt \marginparwidth =
0pt \oddsidemargin = 0pt \sloppy

\usepackage{templateICI}
\usepackage{anysize}
\usepackage{amsmath,amsfonts}
\usepackage{graphicx}
\usepackage{epstopdf}
\usepackage{graphics}
\usepackage[dvips]{epsfig}
\usepackage{times}
\usepackage[latin1]{inputenc}
\usepackage[dvips]{graphicx}
\usepackage[usenames]{color}
\usepackage[spanish]{babel}

\newcommand{\ie}{i.e.}
\newcommand {\out}[1]{}
\newtheorem{teorema}{Teorema}
\newtheorem{lema}{Lema}
\newtheorem{definicion}{Definicion}

\sloppy

\begin{document}
\renewcommand{\listtablename}{�ndice de tablas}
\renewcommand{\tablename}{Tabla}
    \title{\textbf{Desarrollo de una aplicaci�n de anal�tica del
    		aprendizaje basada en posturas corporales
    		utilizando Microsoft Kinect}}
    \author{\textbf{Rodolfo Alejandro Gu��ez Espinoza}}
\principaladviser{Roberto Mu�oz S.}
\coprincipaladviser{NOMBRE DEL CO-REFERENTE}
%\firstreader{NOMBRE DEL INFORMANTE 1}
%\secondreader{NOMBRE DEL INFORMANTE 2}


\beforepreface
\prefacesection{Resumen}
 Coloque aqui un resumen de su trabajo.
\newpage
\prefacesection{Agradecimientos}
Aqui pueden colocar sus agradecimientos.
Si han estudiado con becas es recomendable colocar los agradecimientos a las instituciones que les otorgaron las becas.
\afterpreface

%Aqui deben incluir el fuente de cada capitulo, sin su encabezado.
%\usepackage[spanish]{babel}
\chapter{Introducción}

Este es un ejemplo de como referenciar \cite{bharat97,agrawal94}.
Más ejemplos \cite{beeferman00}. Para más detalle, revise el archivo \textit{template.bib}. 

%\usepackage[spanish]{babel}
\chapter{Marco conceptual}

\section{Lenguaje no verbal}
La comunicación verbal siempre está influenciada por el lenguaje no verbal \cite{bussinessCommunication}. Por telefono el significado del mensaje es afectado por el tono de voz. Presencialmente, es afectado por gestos, expresiones faciales, posturas corporales. Incluso al momento de presentar en alguna charla o conferencia, se utilizan elementos audiovisuales, mapas, diagramas, imágenes, etcétera.

Los elementos más destacados del lenguaje no verbal son: tonos de voz, espacio, sentido, tiempo y el lenguaje corporal.

\section{Lenguaje corporal}



%\section{Tono de voz}
%\section{Espacio}
%\section{Sentido}
%\section{Tiempo}




%\chapter{Estado del arte}

\section{Algoritmos}

\section{Aplicaciones existentes}
\subsection{Anvil}
\subsection{Affectiva}

%\chapter{Definici\'on del problema}

%\chapter{An\'alisis}


%\appendix

%\include{appendix1}
%\include{appendix2}
%\include{appendix3}

\bibliographystyle{plain}
\bibliography{template}

\end{document}
