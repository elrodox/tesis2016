%\usepackage[spanish]{babel}
\chapter{Introducci\'on}

El intercambio de informaci\'on en el aprendizaje es un elemento esencial. Tradicionalmente se pensaba que la comunicaci\'on se desarrollaba s\'olo a trav\'es de la lengua, teniendo as\'i un paradigma monomodal \cite{ref1}. Esto se deb\'ia a que la lengua ocupaba un rol central en la comunicaci\'on, clasificando las otras representaciones como extralingu\'isticas \cite{ref1}. Kress y Leeuwen plantean una postura multimodal, donde el lenguaje est\'a compuesto por distintas aristas: posturas corporales, expresiones faciales, tonos de voz, y un largo etc\'etera \cite{ref13}.
Por ejemplo cuando el profesor ense\~na, tanto \'el como sus alumnos est\'an en una constante entrega y recepci\'on de informaci\'on. A medida que el profesor ense\~na, el alumno va reaccionando, entregando informaci\'on con lenguaje kin\'esico. 

La primera conferencia internacional de Learning Analytics defini\'o la analitica del aprendizaje como ``la medici\'on, recopilaci\'on, an\'alisis y representaci\'on de los datos de los estudiantes y sus contextos, con el prop\'osito de comprender y optimizar el aprendizaje y el entorno en que \'este de lleva a cabo'' \cite{ref5}. Este concepto utiliza un campo muy amplio, incluyendo por ejemplo, \'areas como big data y miner\'ia de datos. Una de las aristas se basa en el lenguaje no verbal, donde el aprendizaje se analiza mediante gestos, posturas corporales, sonidos y otros.

\section{Problema}
El lenguaje no verbal es algo innato que ha estado presente desde hace miles de a\~nos no solo en el comportamiento humano, sino que tambi\'en en los animales. Es algo que surge de manera natural, como por ejemplo cuando una persona habla por tel\'efono y realiza gestos aunque su interlocutor no le est\'e viendo. 
Dado que el aprendizaje es producto de la interacci\'on \cite{ref3}, cuando dos sujetos se comunican, aprenden el uno del otro. Si se estudia este comportamiento, se puede determinar si el receptor est\'a comprendiendo el mensaje recibido o no.
En otras palabras, analizando posturas corporales, se puede determinar la reacci\'on de un individuo a un estimulo externo, una conversaci\'on o un discurso, clasificando su estado en aburrido, distra\'ido, atento y otros m\'as \cite{ref13}.
El proceso de an\'alisis del lenguaje kin\'esico en discursos, reuniones, salas de clases, es una demorosa labor que realizan distintos profesionales de forma "manual" \cite{ref10}. Hoy en d\'ia la anal\'itica del aprendizaje multimodal (MLA) ha avanzado a pasos agigantados en cuanto a publicaciones y literatura \cite{ref2}. Se tienen m\'ultiples investigaciones acerca del tema, ya que el concepto es extenso. Pero en cuanto a desarrollo de software el avance ha sido m\'as lento \cite{ref2}. Por ejemplo se tiene la aplicaci\'on ANVIL \cite{ref11}, la cual toma un archivo de v\'ideo, y realiza un an\'alisis multimodal. Este proyecto no ha sido actualizado desde hace a\~nos, y tiene tecnolog\'ia ya obsoleta \cite{ref2}. ANVIL s\'olo analiza una persona a la vez, lo cual dificulta la tarea a la hora de estudiar varios sujetos. Tambi\'en hay proyectos recientes que se han enfocado en el reconocimiento facial, tonos de voz, y otros aspectos multimodales, de una manera mucho m\'as efectiva. Pero ninguno de estos trabajos realizan un estudio de las posturas est\'aticas de los participantes, adem\'as ejercen el an\'alisis sobre un \'unico sujeto.

\section{Soluci\'on Propuesta}
Seg\'un el problema planteado, se comprende que existe una necesidad de automatizar el proceso de an\'alisis multimodal, lo cual puede tener m\'ultiples ventajas y aplicaciones. Por lo tanto se propone como soluci\'on desarrollar un software que detecte, analice, e interprete mediante posturas corporales est\'aticas, las emociones o estados afectivos de dos sujetos como m\'inimo. 
Para esto se ha decidido utilizar un sensor de profundidad, el cual entregue los puntos en el espacio 3D, que representen el esqueleto de una persona. Dentro de las posibles opciones, se ha optado por utilizar Microsoft Kinect \cite{ref14}, ya que aparte de ser un recurso accesible en la Universidad de Valpara\'iso, existe una gran cantidad de documentaci\'on en internet sobre su utilizaci\'on.
Para el procesamiento, an\'alisis e interpretaci\'on de los puntos detectados y sus respectivas posturas y emociones asociadas, se utilizar\'an m\'aquinas de aprendizaje, para poder realizar esta clasificaci\'on de manera \'optima.
Dentro del marco multimodal hay muchas variantes. Para este trabajo se ha determinado hacer un enfoque espec\'ifico sobre las posturas corporales est\'aticas, identificando un rango aproximado de 3 a 5 estados afectivos.


\section{Objetivos}
\label{obj}
A continuaci\'on se presenta el objetivo general del trabajo de t\'itulo el cual se alcanzar\'a mediante los objetivos espec\'ificos.

\subsection{Objetivo General}
Desarrollar un aplicaci\'on inform\'atica que analice las posturas corporales de los alumnos, e identifique estados de animo en funci\'on del tiempo, con la finalidad de ayudar en el an\'alisis de aprendizaje.

% - Reconocer la emoci\'on que est\'a representando la postura de un sujeto, mediante una aplicaci\'on inform\'atica.

\subsection{Objetivos Espec\'ificos}

\begin{itemize}
	\item Investigar y analizar el dominio del problema, para luego definir el marco de desarrollo del software.
	\item Implementar la aplicaci\'on en base a los resultados del objetivo anterior, realizando pruebas unitarias y mejorando el sistema mediante retroalimentaci\'on.
	\item Integrar el sistema a c\'atedras reales o de prueba en la Universidad, para medir la eficacia final del software.
\end{itemize}
